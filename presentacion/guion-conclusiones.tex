\chapter{Conclusiones y Trabajos Futuros}
\begin{paragraph}{Primer Aporte}
Se realizó un análisis comparativo entre algunas características presentes en los modelos de seguridad de ambas plataformas.

Ellas son: verificación del arranque del sistema operativo,  cifrado del sistema de archivos,  bloqueo del dispositivo  y sistemas de permisos.

Todas ellas se eligieron porque son importantes a la hora de resguardar la privacidad del usuario.
\end{paragraph}
\begin{paragraph}{Segundo Aporte}
Como fruto del análisis, se logró establecer una medida de comparación entre los permisos presentes en ambas plataformas.  La medida propuesta es la siguiente: \emph{dos permisos son similares si resguardan un componente que provee la misma funcionalidad}.

Utilizando la medida propuesta, todos los permisos que un usuario puede cambiar en tiempo de ejecución quedan clasificados en tres grupos:  \emph{Ambas Plataformas},  \emph{Solo en Android}  o \emph{Solo en iOS}. Cabe aclarar que los grupos son mutuamente excluyentes.
\end{paragraph}
\begin{paragraph}{Tercer Aporte}
Otro aporte es el \emph{Framework para la Comparación de Permisos}. Tiene dos funciones principales: determinar empíricamente los alcances de los sistemas de permisos;  y establecer una relación entre los permisos presenten en las dos plataformas.

El \emph{framework} está compuesto por una batería de tests, teniendo cada uno de ellos la tarea de probar una funcionalidad provista por Android e iOS.
\end{paragraph} 
\begin{paragraph}{Cuarto Aporte}
Como resultado de la utilización del \emph{framework} se determinó una clasificación de permisos. El criterio utilizado fue: \emph{un componente pertenece a una clase según requiera autorización explícita del usuario para utilizarlo}.
\end{paragraph}
Para finalizar, se enumeran algunas líneas a seguir como trabajos a futuro:
\begin{itemize}
    \item Probar y mejorar los test que actualmente conforman el framework en dispositivos reales.
    \item Desarrollar tests para las funcionalidades que no pueden ser emuladas (ver Seccion 5.1.1).
    \item Desarrollar un test para poder comparar el cifrado de archivos.
    \item Android otorga todos los permisos \emph{normales}, tal como se enuncia en la Seccion 3.1.4. Pero no sabemos qué permisos otorga iOS. Se podrían desarrollar varios test para descubrirlos.
    \item Dado que salieron al mercado las versiones Android 8.0 e iOS 11, se podría analizar extender el \emph{framework} para las características de seguridad adicionadas en dichas versiones.
    \item En el Capítulo 4 se mencionan algunos análisis previos relacionados a la seguridad de Android y/o de iOS. Se podría profundizar más en comparar el modelo propuesto en el presente informe con los análisis mencionados previamente.
\end{itemize}


\section{Conclusiones y Trabajos Futuros}
\begin{frame}
 \begin{center}
  \LARGE Conclusiones\\ y\\ Trabajos Futuros
 \end{center}
\end{frame}
\begin{frame}
 \frametitle{Conclusiones}
 \begin{footnotesize}
 \begin{exampleblock}{Primer Aporte}
Se realizó un análisis comparativo entre algunas características presentes en los modelos de seguridad de ambas plataformas.\\ \pause
Ellas son: verificación del arranque del sistema operativo, \pause cifrado del sistema de archivos, \pause bloqueo del dispositivo \pause y sistemas de permisos.\\ \pause
Todas ellas se eligieron porque son importantes a la hora de resguardar la privacidad del usuario.
 \end{exampleblock}\pause
 \begin{exampleblock}{\flushright {Segundo Aporte}}
Como fruto del análisis, se logró establecer una medida de comparación entre los permisos presentes en ambas plataformas. \pause La medida propuesta es la siguiente: \emph{dos permisos son similares si resguardan un componente que provee la misma funcionalidad}.\\ \pause
Utilizando la medida propuesta, todos los permisos que un usuario puede cambiar en tiempo de ejecución quedan clasificados en tres grupos: \pause \emph{Ambas Plataformas}, \pause \emph{Solo en Android} \pause o \emph{Solo en iOS}.\\ \pause
Cabe aclarar que los grupos son mutuamente excluyentes.
 \end{exampleblock}
 \end{footnotesize}
\end{frame}
\begin{frame}
 \frametitle{Conclusiones}
 \begin{footnotesize}
 \begin{exampleblock}{Tercer Aporte}
Otro aporte es el \emph{Framework para la Comparación de Permisos}. \pause Tiene dos funciones principales: \pause determinar empíricamente los alcances de los sistemas de permisos; \pause y establecer una relación entre los permisos presenten en las dos plataformas.\\ \pause
El \emph{framework} está compuesto por una batería de tests, teniendo cada uno de ellos la tarea de probar una funcionalidad provista por Android e iOS.
 \end{exampleblock} \pause
 \begin{exampleblock}{\flushright {Cuarto Aporte}}
Como resultado de la utilización del \emph{framework} se determinó una clasificación de permisos. \pause El criterio utilizado fue: \emph{un componente pertenece a una clase según requiera autorización explícita del usuario para utilizarlo}. \pause Utilizando el criterio propuesto, se obtuvieron cuatro clases mutuamente excluyentes:\pause
  \begin{scriptsize}
  \begin{itemize}[<+->]
    \item \underline{Clase A}: componentes que requieren autorización explícita en ambas plataformas para poder utilizar las funcionalidades que proveen;
    \item \underline{Clase B}: componentes que requieren autorización explícita solamente en Android;
    \item \underline{Clase C}: componentes que requieren autorización explícita solamente en iOS;
    \item \underline{Clase D}: componentes que no requieren autorización explícita para poder utilizar las funcionalidades que proveen.
  \end{itemize}
  \end{scriptsize}
 \end{exampleblock}
 \end{footnotesize}
\end{frame}
\begin{frame}
 \frametitle{Trabajos futuros}
Para finalizar, se enumeran algunas líneas a seguir como trabajos a futuro:\pause
 \begin{small}
 \begin{itemize}[<+->]
    \item Probar y mejorar los tests en dispositivos reales.
    \item Desarrollar tests para las funcionalidades que no pueden ser emuladas.
    \item Desarrollar un test para poder comparar el cifrado de archivos.
    \item No sabemos qué permisos básicos otorga iOS. Se podrían desarrollar varios tests para descubrirlos.
    \item Dado que salieron al mercado Android 8.0 e iOS 11, se podría analizar extender el \emph{framework} para las características de seguridad adicionadas en dichas versiones.
    \item En el Capítulo 4 se mencionan algunos análisis previos relacionados a la seguridad de Android y/o de iOS. Se podría profundizar más en comparar el modelo propuesto en el presente informe con los análisis mencionados previamente.
 \end{itemize}
 \end{small}
\end{frame}


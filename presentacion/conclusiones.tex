\section{Conclusiones y Trabajos Futuros}
\begin{frame}
 \begin{center}
  \LARGE Conclusiones\\ y\\ Trabajos Futuros
 \end{center}
\end{frame}
\begin{frame}
 \frametitle{Conclusiones}
 \begin{footnotesize}
 \begin{exampleblock}{Primer Aporte}
Un análisis comparativo entre algunas características presentes en los modelos de seguridad de ambas plataformas.
 \end{exampleblock}\pause
 \begin{exampleblock}{\flushright {Segundo Aporte}}
Se logró establecer una clasificación de todos los permisos que un usuario puede cambiar en tiempo de ejecución.\\ \pause La medida propuesta es la siguiente: \emph{dos permisos son similares si resguardan un componente que provee la misma funcionalidad}.\\ 
 \end{exampleblock}\pause
 \begin{exampleblock}{Tercer Aporte}
El \emph{Framework para la Comparación de Permisos} multiplataforma.
 \end{exampleblock} \pause
 \begin{exampleblock}{\flushright {Cuarto Aporte}}
Se determinó una clasificación de permisos.\\ \pause El criterio utilizado fue: \emph{un componente pertenece a una clase según requiera autorización explícita del usuario para utilizarlo}.
 \end{exampleblock}
 \end{footnotesize}
\end{frame}
\begin{frame}
 \frametitle{Trabajos futuros}
Para finalizar, se enumeran algunas líneas a seguir como trabajos a futuro:\pause
 \begin{small}
 \begin{itemize}[<+->]
    \item Probar y mejorar los tests en dispositivos reales.
    \item Desarrollar tests para las funcionalidades que no pueden ser emuladas.
    \item Desarrollar un test para poder comparar el cifrado de archivos.
    \item No sabemos qué permisos básicos otorga iOS. Se podrían desarrollar varios tests para descubrirlos.
    \item Dado que salieron al mercado Android 8.0 e iOS 11, se podría analizar extender el \emph{framework} para las características de seguridad adicionadas en dichas versiones.
    \item En el Capítulo 4 se mencionan algunos análisis previos relacionados a la seguridad de Android y/o de iOS. Se podría profundizar más en comparar el modelo propuesto en el presente informe con los análisis mencionados previamente.
 \end{itemize}
 \end{small}
\end{frame}


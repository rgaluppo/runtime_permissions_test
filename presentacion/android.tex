\subsection{Modelo de Android}
\begin{frame}
 \begin{center}
  \LARGE Modelo de Android
 \end{center}
\end{frame}
\begin{frame}
 \frametitle{Modelo de Android}
 \begin{figure}[tH]
  \begin{subfigure}{0.58\linewidth}
  \begin{small}
   \begin{itemize}
    \item Android es un sistema operativo de código abierto, diseñado para dispositivos móviles y desarrollado por Google junto con la Open Handset Alliance.
    \item <3->{Cada aplicación corre en un \emph{entorno aislado}, forzando a que solo pueda tener acceso a sus recursos.}
   \end{itemize}
     \end{small}
  \end{subfigure}
  \begin{subfigure}{0.4\linewidth}\pause
    \centering
   		\includegraphics[width=\linewidth]{android-stack}
  \end{subfigure}
  \caption{Capas de Android.}
\end{figure}
\end{frame}
\begin{frame}
 \frametitle{Modelo de Android}
 \begin{small}
 \begin{itemize}
     \item Podemos clasificar los permisos según el riesgo implícito al otorgarlos:
     \begin{multicols}{2}
     \begin{itemize}[<+->]\small
      \item \emph{\textbf<5->{Normal}}
      \item \emph{\textbf<5->{Dangerous}}
      \item \emph{\invisible<5->{Signature}}
      \item \emph{\invisible<5->{Signature/System}}
     \end{itemize}
     \end{multicols}\pause
     \item A partir de la versión 6.0 se propone un nuevo modelo de permisos:\pause
         \begin{figure}[btp]
            \begin{subfigure}{0.4\linewidth}
            \centering
                \includegraphics[width=.5\linewidth]{allow_contact}
                \caption{Solicitud de un permiso.}
            \end{subfigure}
        \begin{subfigure}{0.4\linewidth}\pause
        \centering
            \includegraphics[width=.5\linewidth]{app-permissions}
            \caption{Listado de los permisos.}
    	\end{subfigure}
    	\caption{Nuevo modelo de permisos.}
        \end{figure}
 \end{itemize}
  \end{small}
\end{frame}

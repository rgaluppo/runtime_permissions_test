\chapter{Hacia un Framework Comparativo}
Android e iOS permiten cambiar ciertos permisos de una aplicación en tiempo de ejecución, es decir, luego de haberla instalado en el dispositivo. Para poner a prueba los sistemas de permisos de ambas plataformas, se ha desarrollado un framework que permite comparar formalmente diversos aspectos de seguridad.\\
En las siguientes secciones se detallaran los distintos tests que componen el framework. Además, se mencionarán las conclusiones arribadas luego de correr los tests mencionados anteriormente.
\section{Vista principal}
El framework es una aplicación móvil y esta compuesto por varios tests. Cada test pone a prueba a un componente del dispositivo, permitiendo así conocer el alcance de los permisos correspondientes a dicho componente.\\
Todos los test fueron implementados en \emph{JavaScript} y corren en el dispositivo mediante \emph{Apache Cordova}\footnote{TODO: va la referencia hacia donde se explica que es Apache Cordova.}.\\
Al iniciar la aplicación, lo primero que se observa son observan dos áreas principales: Acciones y Test.\\
La primera área contiene un botón para acceder a la configuración de los permisos del dispositivo. Allí, el \textit{tester} puede cambiar manualmente los permisos requeridos por la aplicación. Además, se encuentra un botón para limpiar la consola (que se encuentra en el otro área).\\
Mientras que la segunda área se subdivide en dos: en la parte de los tests y la parte de la consola. Una parte corresponde a los botones de los tests que, al presionarse, ejecutan el respectivo test, mostrando en la consola el resultado. Dicho resultado se mostrara con tipografía color verde si fue exitoso; en cambio, se mostrara con tipografía color roja de ser fallido.\\
\begin{figure}[hbtp]
    \centering
	\includegraphics[width=0.4\linewidth]{chapter5/app_main_view}
	\caption{Áreas de la aplicación \textit{Runtime Permissions Test}.}
	\label{fig:chapter05:main_view}
\end{figure}
En la Figura \ref{fig:chapter05:main_view} se observan las áreas del framework.\\

A continuación se mencionan los componentes que se pueden testear con el framework:
\begin{itemize}
	\item Contactos
	\item Calendario
	\item Geolocalización
	\item SMS
	\item Sensores
	\item Almacenamiento
	%\item AppAvaiable (funciona en ambos) Es medio tonto. Probar si detecta mi app...
	\item Información del dispositivo.
	\item Acceso a Internet
	%\item Bateria (falta)
	%\item Health (falta - solo en iOS)
\end{itemize}
\subsection{Funciones no compatibles}
El emulador ofical de Android es compatible con la mayoría de las funciones de un dispositivo, pero no incluye la posibilidad de virtualizar los siguientes componentes \cite{daemu}:
\begin{itemize}
    \item WiFi;
    \item Bluetooth;
    \item NFC\footnote{Del ingles \emph{Near Field Communication}. Es una tecnología de comunicación inalámbrica, de corto alcance y alta frecuencia que permite el intercambio de datos entre dispositivos.};
    \item Manipulación de la tarjeta SD;
    \item Conexión USB;
    \item Micrófono;
    \item Cámara
\end{itemize}
Al no poder manipular la tarjeta SD, no es posible testear ninguna las funcionalidades multimedia: no se pueden grabar audio, ni video ni sacar fotos.\\
Por lo tanto, no se agregaron al framework tests para los componentes listados anteriormente.
\section{C\'atalogo de test}
En esta sección se listaran todos los test que conforman al framework. Para cada test se detallara su algoritmo, los plugins de Apache Cordova que se utilizaron para confeccionarlo y una serie de capturas.\\
Para acceder al panel de configuraciones, se utilizo el siguiente plugin:\\ \href{https://www.npmjs.com/package/cordova.plugins.diagnostic}{cordova.plugins.diagnostic v3.1.7}
\subsection{Contactos}
El test consiste en crear un contacto y luego listar todos los contactos presentes en el dispositivo. En caso de ser exitoso, se muestran los contactos. De lo contrario, se muestra un error.\\
\begin{algorithm}
	\begin{algorithmic}[1]
		\STATE se listan todos los contactos en la consola.
		\STATE se crea un nuevo contacto.
		\STATE se listan todos los contactos en la consola.
	\end{algorithmic}
	\caption{Test de Contactos.}\label{alg:chap5:test_contactos}
\end{algorithm}
\textbf{\emph{Plugin:}} \href{https://www.npmjs.com/package/cordova-plugin-contacts}{cordova-plugin-contacts v2.2.1}
\begin{figure}[hbtp]
    \centering
    \begin{subfigure}{0.3\linewidth}
        \includegraphics[width=\linewidth]{chapter5/without_contact}
        \label{fig:chapter05:without_contact}
    \end{subfigure}
    \begin{subfigure}{0.3\linewidth}
        \includegraphics[width=\linewidth]{chapter5/with_contact}
        \label{fig:chapter05:with_contact}
    \end{subfigure}
    \caption{Testeando la administración de los contactos.}
	\label{fig:ch05:contacts-cases}
\end{figure}
\newpage
\subsection{Calendario}
El test consiste en crear un evento en un determinado rango de fechas y luego listar todos los eventos dentro del rango. En caso de ser exitoso, se muestran los datos del evento. De lo contrario, se muestra un error.\\
\begin{algorithm}
	\begin{algorithmic}[1]
		\STATE se crea la fecha $startDate$
		\STATE se crea la fecha $endDate$
		\STATE se crea un evento que empieza en la fecha $startDate$ y termina en la fecha $endDate$.
		\STATE se listan en la consola los eventos entre las fechas $startDate$ y $endDate$.
	\end{algorithmic}
	\caption{Test del Calendario.}\label{alg:chap5:test_calendario}
\end{algorithm}
\textbf{\emph{Plugin:}} \href{https://www.npmjs.com/package/cordova-plugin-calendar}{cordova-plugin-calendar v4.5.5}
\begin{figure}[hbtp]
    \centering
    \begin{subfigure}{0.3\linewidth}
        \includegraphics[width=\linewidth]{chapter5/without_calendar}
        \label{fig:ch05:without_calendar}
    \end{subfigure}
    \begin{subfigure}{0.3\linewidth}
        \includegraphics[width=\linewidth]{chapter5/with_calendar}
        \label{fig:ch05:with_calendar}
    \end{subfigure}
    \caption{Caso exitoso y caso fallido.}
	\label{fig:ch05:calendar-cases}
\end{figure}
\newpage
\subsection{Geolocalización}
Para realizar este test, se configuro el emulador de Android para que simule las coordenadas \texttt{(-122$^\circ$, 37$^\circ$)}, tal como se observa en la Figura \ref{fig:ch05:android_extended_controls}. Dicha configuración también se realizo en emulador oficial de iOS.\\
\begin{figure}[hbtp]
    \centering
	\includegraphics[width=0.6\linewidth]{chapter5/android_extended_controls}
	\caption{Panel de configuración de coordenadas del emulador de Android.}
	\label{fig:ch05:android_extended_controls}
\end{figure}
\begin{algorithm}
	\begin{algorithmic}[1]
		\STATE Se censa el GPS.
		\STATE Se muestran los datos en la consola.
	\end{algorithmic}
	\caption{Test de Geolocalización.}\label{alg:chap5:test_geolocalizacion}
\end{algorithm}
\textbf{\emph{Plugin:}} \href{https://github.com/apache/cordova-plugin-geolocation}{cordova-plugin-geolocation v2.4.3}
\begin{figure}[hbtp]
    \centering
	\begin{subfigure}{0.3\linewidth}
		\includegraphics[width=\linewidth]{chapter5/without_location}
		\label{fig:ch05:without_location}
	\end{subfigure}
	\begin{subfigure}{0.3\linewidth}
		\includegraphics[width=\linewidth]{chapter5/location_success}
		\label{fig:ch05:with_location}
	\end{subfigure}
	\caption{Testeando la geolocalización.}
	\label{fig:ch05:geolocation-cases}
\end{figure}
\newpage
\subsection{SMS}
En un principio, se diseño un test con la capacidad de leer los mensajes. Al momento de implementarlo, se encontró una restricción de seguridad en iOS: a partir de la version 8 no se pueden acceder a dichos mensajes desde una aplicación instalada por el usuario \cite{foda}. En cambio, en Android si se pueden acceder, siempre que se tengan los permisos correspondientes.\\
Es por ello que se decidió quitar dicha funcionalidad, quedando disponible únicamente la posibilidad de enviar mensajes SMS.\\
\begin{algorithm}
	\begin{algorithmic}[1]
		\STATE Se inicializan los eventos para recibir SMS. 
		\STATE Se envía un SMS de prueba.
		\STATE Se indica el resultado del test en la consola.
	\end{algorithmic}
	\caption{Test de SMS.}\label{alg:chap5_test_sms}
\end{algorithm}
\textbf{\emph{Plugin:}} \href{https://github.com/floatinghotpot/cordova-plugin-sms}{cordova-plugin-sms v.1}
\begin{figure}[hbtp]
    \centering
	\begin{subfigure}{.3\linewidth}
		\includegraphics[width=\linewidth]{chapter5/without_sms}
		\label{fig:ch05:without_sms}
	\end{subfigure}
	\begin{subfigure}{.3\linewidth}
	    \centering
		\includegraphics[width=\linewidth]{chapter5/success_sms}
		\label{fig:ch05:with_sms}
	\end{subfigure}
	\caption{Test de los permisos de los mensajes SMS.}
	\label{fig:chapter05:sms_test}
\end{figure}
\newpage
\subsection{Sensores}
El objetivo de este test es obtener datos de dos sensores del dispositivo: del acelerómentro y del giroscopio. Para ello, se configura un \texttt{timer}. Durante el tiempo que este activo, se tomaran distintos muestreos; y cuando ocurra el \emph{timeout} se mostraran los datos en la consola de la aplicación.\\
\begin{algorithm}
	\begin{algorithmic}[1]
		\STATE Se inicializa un TIMER con \texttt{5 seg} para detener las mediciones.
		\STATE Se inicia la medición del acelerómentro.
		\STATE Se inicia la medición del giroscopio.
		\STATE Se muestran los resultados en la consola.
	\end{algorithmic}
	\caption{Test de los Sensores.}\label{alg:chap5_test_sensors}
\end{algorithm}
\textbf{\emph{Plugin:}} \href{https://www.npmjs.com/package/cordova-plugin-device-motion}{cordova-plugin-device-motion}\\
\textbf{\emph{Plugin:}} \href{https://www.npmjs.com/package/cordova-plugin-gyroscope}{cordova-plugin-gyroscope}\\
\begin{figure}[hbtp]
    \centering
	\begin{subfigure}{.3\linewidth}
	    \centering
		\includegraphics[width=\linewidth]{chapter5/accelerometer_success}
		\label{fig:ch05:accelerometer_success}
	\end{subfigure}
	\caption{Datos medidos.}
	\label{fig:chapter05:sensors_test}
\end{figure}
En iOS no fueron necesarios permisos para poder correr el test.
\newpage
\subsection{Almacenamiento}
El presente test fue diseñado para probar el alcance de los permisos de escritura sobre el sistema de archivos que tiene cada plataforma.\\
\begin{algorithm}
	\begin{algorithmic}[1]
		\STATE Se intenta capturar la pantalla.
		\STATE Se guarda la captura en el dispositivo.
		
	\end{algorithmic}
	\caption{Test de Almacenamiento.}\label{alg:chap5_test_storage}
\end{algorithm}
\textbf{\emph{Plugin:}} \href{https://github.com/gitawego/cordova-screenshot}{cordova-screenshot}\\
\begin{figure}[hbtp]
   \centering
   	\begin{subfigure}{.3\linewidth}
		\includegraphics[width=\linewidth]{chapter5/storage_fail}
		\label{fig:ch05:storage_fail}
	\end{subfigure}
	\begin{subfigure}{.3\linewidth}
		\includegraphics[width=\linewidth]{chapter5/storage_success}
		\label{fig:ch05:storage_success}
	\end{subfigure}
	\caption{Testeando el almacenamiento en el dispositivo.}
	\label{fig:ch05:storage_test}
\end{figure}
En iOS no fueron necesarios permisos para poder correr el test.
\newpage
\subsection{DeviceInfo}
El objetivo de este test es obtener datos del dispositivo donde corre la aplicación: plataforma, modelo, serial, entre otros.\\
\begin{algorithm}
	\begin{algorithmic}[1]
		\STATE Se obtienen los datos del dispositivo.
		\RETURN un \texttt{tag} \textless IMG\textgreater, cuyo \texttt{source} es la captura realizada.
	\end{algorithmic}
	\caption{Test de Informacion del Dispositivo.}\label{alg:chap5_test_info}
\end{algorithm}
\textbf{\emph{Plugin:}} \href{https://www.npmjs.com/package/cordova-plugin-device}{cordova-plugin-device}\\
\begin{figure}[hbtp]
   \centering
   	\begin{subfigure}{.3\linewidth}
		\includegraphics[width=\linewidth]{chapter5/device_info}
		\label{fig:ch05:device-info-success}
	\end{subfigure}
	\caption{Obteniendo datos del dispositivo.}
	\label{fig:ch05:device-info}
\end{figure}
No fueron necesarios permisos en ninguna de las dos plataformas para poder correr el test.
\newpage
\subsection{Internet}
El objetivo del presente test es establecer una comunicación a traves de Internet. No se requirió de ningún plugin para implementarlo.\\
La decodificación de la imagen se obtuvo de \href{https://stackoverflow.com/questions/19124701/get-image-using-jquery-ajax-and-decode-it-to-base64/25371174#25371174}{StackOverflow}. Al ejecutarse en un emulador, para probar el acceso a Internet, se habilitó/deshabilitó la Red Inalámbrica de la computadora donde se corrieron los emuladores.\\
\begin{algorithm}
	\begin{algorithmic}[1]
		\STATE se realiza una consulta GET HTTP hacia \href{https://dcc.fceia.unr.edu.ar/sites/all/themes/birthofcool/images/logo-lcc.png}{logo del DCC}
		\STATE se decodifica la imagen (viene codificada en Base64).
		\RETURN un \texttt{tag} \textless IMG\textgreater, cuyo \texttt{source} es el dato decodificado.
	\end{algorithmic}
	\caption{Test de conexión a Internet.}\label{alg:chap5_test_internet}
\end{algorithm}
\begin{figure}[hbtp]
    \centering
    \begin{subfigure}{.3\linewidth}
		\includegraphics[width=\linewidth]{chapter5/fail_request}
		\label{fig:ch05:fail_request}
	\end{subfigure}
	\begin{subfigure}{.3\linewidth}
		\includegraphics[width=\linewidth]{chapter5/success_request}
		\label{fig:ch05:success_request}
	\end{subfigure}
	\caption{Testeando el acceso a Internet.}
	\label{fig:ch05:internet_test}
\end{figure}
No fueron necesarios permisos en ninguna de las dos plataformas para poder correr el test.
\section{Resultados experimentales}

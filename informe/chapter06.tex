\chapter{Conclusiones y Trabajos Futuros}
Android Marshmallow incorpora a una novedosa manera de administrar el sistema de permisos: los permisos \emph{peligrosos} se pueden otorgar o denegar en cualquier momento; mientras que iOS ya tenía una forma similar desde versiones anteriores. Si bien son muchos los trabajos que comparan los modelos de seguridad de ambas plataformas, el presente informe realiza aportes basado en un análisis sobre el sistema de permisos.\\

Como primer aporte, en el capítulo 3 se realiza un análisis comparativo entre algunas características presentes en los modelos de seguridad de ambas plataformas. Ellas son: verificación del arranque del sistema operativo, cifrado del sistema de archivos, bloqueo del dispositivo y sistemas de permisos. Todas ellas se eligieron porque son importantes a la hora de resguardar la privacidad del usuario. Es por ello que la característica analizada con mayor profundidad fue el sistema de permisos. Como fruto del análisis, se logro establecer una medida de comparación entre los permisos presentes en ambas plataformas. La medida propuesta es la siguiente: \emph{dos permisos son similares si resguardan un componente que provee la misma funcionalidad}. Por ejemplo, ambas plataformas tienen un componente que permite obtener la localización del dispositivo. Además, dichos componentes tienen un permiso, en cada plataforma, que lo resguarda. Por lo tanto, dichos permisos son similares. Utilizando la medida propuesta, todos los permisos que un usuario puede cambiar en tiempo de ejecución, quedan clasificados en tres grupos: \emph{Ambas Plataformas}, \emph{Solo en Android} o \emph{Solo en iOS}. Cabe aclarar que los grupos son mutuamente excluyentes. El resultado de la clasificación se observa en el Cuadro \ref{tab:chapter03:compPerm}.\\ 

Otro aporte del presente informe es el \emph{framework} propuesto en el capítulo anterior. Tiene dos funciones principales: determinar empíricamente los alcances de los sistemas de permisos; y establecer una relación entre los permisos presenten en las dos plataformas. Para ello, está compuesto por una batería de tests, teniendo cada uno de ellos la tarea de probar una funcionalidad provista por Android e iOS. Como resultado de la utilización del \emph{framework} propuesto, se encontró una clasificación de permisos. El criterio utilizado para clasificarlos fue el siguiente: \emph{un componente pertenece a una clase según requiera autorización explícita del usuario para utilizarlo}. Utilizando el criterio propuesto, se obtienen cuatro clases mutuamente excluyentes:
\begin{itemize}
    \item \underline{Clase A}: componentes que requieren autorización explícita en ambas plataformas para poder utilizar las funcionalidades que proveen;
    \item \underline{Clase B}: componentes que requieren autorización explícita solamente en Android;
    \item \underline{Clase C}: componentes que requieren autorización explícita solamente en iOS;
    \item \underline{Clase D}: componentes que no requieren autorización explícita para poder utilizar las funcionalidades que proveen.
\end{itemize}
Se pueden mencionar varias observaciones. La primera de ellas es la clasificación de varios componentes, no solamente los que permiten cambiar su permiso en tiempo de ejecución. Esto se observa en el Cuadro \ref{tab:ch03:permission-classification}. Otra observación es que, todas las clases contienen al menos un elemento, salvo la \emph{Clase C}. Sin embargo, se mantuvo en el informe ya que, en futuras versiones del \emph{framework}, es posible que pueda ser descubierto algún miembro de ella. Como última observación, algunos de los miembros de la \emph{Clase D} son importantes a la hora de resguardar la privacidad del usuario. Un ejemplo de esto, es el permiso de \emph{Acceso a Internet}. Si bien solo no vulnera la privacidad, combinado con algún otro, deja expuesta información sensible. Es por ello que, algunos miembros de la \emph{Clase D} deberían pertenecer a la \emph{Clase A}.\\

Para finalizar, se mencionan los trabajos a futuro. Una alternativa posible es incrementar las capacidades del \emph{framework} presentado en el capítulo anterior. Al ser desarrollado en la plataforma Apache Cordova, se puede extender fácilmente. A continuación, se enumeran las direcciones de crecimiento:
\begin{itemize}
    \item Probar y mejorar los test que actualmente conforman el framework en dispositivos reales.
    \item Desarrollar tests para las funcionalidades que no pueden ser emuladas (ver Seccion 5.1.1, \cite{foda, foda2}).
    \item Desarrollar un test para poder comparar el cifrado de archivos.
    \item Android otorga todos los permisos \emph{normales}, tal como se enuncia en la Seccion 3.1.4. Pero no sabemos qué permisos otorga iOS. Se podrían desarrollar varios test para descubrirlos.
    \item Dado que salieron al mercado las versiones Android 7.0 e iOS 10, se podría analizar extender el \emph{framework} para las características de seguridad adicionadas en dichas versiones.
\end{itemize}
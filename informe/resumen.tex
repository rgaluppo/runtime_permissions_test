\begin{abstract}
La seguridad en dispositivos móviles se ha convertido en un asunto muy importante debido al incremento de ataques recibidos y a las consecuencias que estos tienen. Los ataques vienen incentivados por la popularización de los
dispositivos móviles, el aumento de información personal y confidencial que almacenan y las operaciones realizadas a través de ellos, como por ejemplo las bancarias. Es por ello que es de vital importancia analizar sus respectivos modelos de seguridad, con el objetivo de encontrar similitudes y diferencias.\\

Este trabajo realiza aportes basado en un análisis detallado sobre las características de seguridad en Android e iOS, para sus versiones más avanzadas, con el objetivo de preservar la privacidad del usuario. En particular, considerando permisos que se pueden modificar en tiempo de ejecución. Sumado al análisis, se presenta un \emph{framework} comparativo. Es una aplicación móvil híbrida, que puede ejecutarse tanto en Android como en iOS. Tiene dos funciones principales: determinar empíricamente los alcances de los sistemas de permisos; y establecer una relación entre los permisos presenten en ambas plataformas.
\end{abstract}
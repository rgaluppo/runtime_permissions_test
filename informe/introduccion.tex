\chapter{Introducción}
En los últimos años se ha observado un marcado incremento en el número de dispositivos móviles
que tienen a iOS y Android como sistemas operativos. Actualmente, más del 99\% de los dispositivos móviles en el mercado tiene uno de estos sistemas operativos \cite{wss}. Hay muchas aplicaciones que se desarrollan para estas dos plataformas. El número actual de aplicaciones de Android en el mercado supera los 3.500.00 \cite{GPS} y para iOS asciende a más de 3.100.000 \cite{ASM}. Debido al uso diario de estas aplicaciones, se puede filtrar una gran cantidad de información privada y confidencial a menos que se aplique control de acceso a las aplicaciones instaladas.\\

Android \cite{aos} es un sistema operativo de código abierto \cite{aosp}, diseñado para dispositivos móviles y desarrollado por Google junto con la Open Handset Alliance \cite{oha}. Su modelo de seguridad, a pesar de poseer características muy variadas, presenta ciertas limitaciones. Algunos trabajos previos dan cuenta de la rigidez del sistema de permisos a la hora de, por ejemplo, instalar una nueva aplicación. Además, muchos de los aspectos de la seguridad en Android dependen de la correcta construcción de las aplicaciones por parte de los desarrolladores y, al mismo tiempo, no existe una documentación precisa que facilite dicha tarea.\\

Por otra parte, iOS es un sistema operativo móvil de la multinacional Apple Inc. diseñado para ser seguro \cite{asg}. Cada dispositivo combina \emph{hardware}, \emph{software} y servicios, diseñados para trabajar conjuntamente para proveer seguridad y al mismo tiempo, que la misma sea transparente para el usuario. Las principales características de seguridad, como el cifrado del dispositivo, no son configurables y vienen habilitadas por defecto, por lo que los usuarios no pueden deshabilitarlas por error. La seguridad se extiende más allá del dispositivo, incluido todo lo que hacen los usuarios localmente, en redes y con servicios clave de Internet. Como consecuencia de ello, se genera un ecosistema seguro.\\

Analizar detalladamente las características de seguridad en ambos sistemas, para sus versiones más avanzadas, permitirá entender las fortalezas y debilidades comparativas de cada uno. Existen muchas formas de comparación posibles. Por ejemplo, la medida propuesta en \cite{YA2014} consiste en analizar la seguridad de una aplicación móvil en cada fase del ciclo de vida, comparándola en cada plataforma. En \cite{HYGZD2014} el enfoque es distinto; se centra en comparar los permisos requeridos a cada plataforma al momento de instalar aplicaciones presentes en ambos sistemas. En \cite{Gor16, BCLR15, Rom14} se cambia el enfoque propuesto. El objetivo que persiguen estos trabajos es desarrollar una especificación formal que describa el modulo de seguridad de Android. En cambio, en \cite{TZSH13}, el enfoque se centra en distintos ataques al modelo de seguridad de iOS. En el presente trabajo se analizarán formas alternativas de comparación. En particular, considerando permisos que se pueden modificar en tiempo de ejecución.\\

A continuación se describe la organización del trabajo. En en el segundo y tercer capítulo se presentan los modelos de seguridad de Android e iOS, respectivamente. El análisis comparativo de ambos sistemas operativos se realiza en el cuarto capítulo. Luego, en el quinto capítulo se introduce la plataforma de desarrollo Apache Cordova. Con ella, se realizó un \emph{framework} que permite comparar empíricamente los sistemas de permisos de Android e iOS. Los resultados de dicha comparación, junto con la composición del \emph{framework}, se encuentran en el sexto capítulo. Finalmente, en el séptimo y último capítulo se detallan las conclusiones y los trabajos futuros.
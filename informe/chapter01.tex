\chapter{Modelo de Seguridad de Android}
Android \cite{aos} es un sistema operativo de código abierto, diseñado para dispositivos móviles y desarrollado por Google junto con la Open Handset Alliance. Su arquitectura sigue el estilo arquitectónico conocido como Sistemas Estratificados: los distintos componentes se agrupan en capas según su nivel de abstracción, conformando una jerarquía. Las capas inferiores contienen componentes ligados al \textit{hardware}, mientras que las capas superiores agrupan componentes ligados con tareas de más alto nivel.\\

Una de las características principales de Android es que cualquier aplicación, ya sea principal o
creada por algún desarrollador, puede, al instalarse con las autorizaciones adecuadas, utilizar tanto
los recursos/servicios del dispositivo móvil como los ofrecidos por el resto de las aplicaciones
instaladas.\\

A lo largo de este capítulo se presentará una descripción de los principales aspectos del modelo de seguridad de Android,  en la versión 6.0 llamada Marshmallow, lanzada en 2015 \cite{aosm}.
\section{Entorno aislado para cada aplicación}\label{ch01-sandbox}
Fue una de las primeras tecnologías de seguridad aplicadas en Android, y tiene mucha importancia en el modelo de seguridad. Consiste en que cada aplicación se ejecuta en un \emph{entorno aislado}\footnote{Traducción propuesta para el término \textit{sandbox}.}, forzando a que solo pueda tener acceso irrestricto a sus propios recursos. Por lo tanto, las aplicaciones no pueden interactuar entre ellas y tienen acceso limitado al sistema operativo. A cada aplicación se le asigna un único id de usuario (UID) y se ejecuta a nombre de ese usuario como un proceso independiente, tal como se observa en la Figura \ref{fig:ch01:sandbox}.\\

Dado que el \emph{entorno aislado} es a nivel del \textit{kernel}, este modelo de seguridad se extiende al código nativo y a las aplicaciones del sistema operativo, tales como las bibliotecas del sistema operativo y los \textit{frameworks} de las aplicaciones. Esto genera un aislamiento a nivel del \textit{kernel}, ya que todas las políticas que se aplican a usuarios o grupos de usuarios se transfieren a las aplicaciones por tener su UID.
\begin{figure}[hbtp]
	\begin{center}
		\includegraphics[width=\linewidth]{chapter1/android_security_model}
	    \caption{Aislamiento de las aplicaciones según su UID \cite{asreview2015}.}
	    \label{fig:ch01:sandbox}
    \end{center}
\end{figure}
\section{Políticas de acceso mejoradas}
Como parte de su modelo de seguridad, Android utiliza \emph{SELinux}\footnote{\textit{Security Enhanced Linux}, por sus siglas en inglés.} para hacer cumplir el control de acceso obligatorio (MAC) sobre todos los procesos. A partir de Android 4.3, se utiliza \emph{SELinux} para definir los límites del \emph{entorno aislado} de cada la aplicación \cite{aosel}.\\

\emph{SELinux} es un sistema de políticas de acceso obligatorio para Linux. Por lo tanto, siempre se consulta a una autoridad central para permitir cualquier acceso a un recurso; abarca a todos los procesos, inclusive aquellos que corren con privilegios de \textit{root}. \emph{SELinux} opera bajo \textit{the ethos of default denial}, es decir, todo lo que no está explícitamente permitido es denegado.\\

\emph{SELinux} depende de las etiquetas para unir acciones y políticas. Las etiquetas determinan lo que está permitido. Los sockets, archivos y procesos tienen etiquetas en \emph{SELinux}. Decimos que un dominio es un conjunto de procesos en una política de seguridad. Cada dominio lleva un nombre que denominamos etiqueta de dominio.  Las decisiones de \emph{SELinux} se basan fundamentalmente en las etiquetas asignadas a estos objetos y la política que define cómo pueden interactuar. Es por ello que los procesos que comparten una misma etiqueta de dominio son tratados de forma idéntica por la política de seguridad.\\

Android asigna un rol, un nombre de usuario y un dominio al usuario. Por lo tanto, aunque varios usuarios pueden compartir el mismo nombre de usuario, el control de acceso se administra a través del dominio, que está configurado por diferentes políticas. Estas políticas generalmente incluyen instrucciones y permisos específicos, que el usuario debe poseer para obtener acceso al sistema.\\

\emph{SELinux} puede trabajar en dos modos \cite{aosel}:
\begin{itemize}
    \item \underline{Riguroso:} en el que los rechazos de permisos se registran y se aplican.
    \item \underline{Permisivo:} las denegaciones de permisos se registran pero no se aplican.
\end{itemize} 
Android permite aplicar el modo permisivo en un determinado dominio y el resto del sistema permanece en modo riguroso. Gracias a ello, se puede lograr aplicar incrementalmente \emph{SELinux} a una porción cada vez mayor del sistema y desarrollar de políticas para nuevos servicios, manteniendo el resto del sistema en vigencia.
\section{Seguridad en las aplicaciones}
\subsection{Permisos} \label{ch01-permisos}
Ciertos recursos que provee Android son sensibles, ya que acceden a datos personales o periféricos importantes. Dichos recursos sólo pueden ser accedidos mediante una SS-API\footnote{\textit{Security Sensitive API}, por sus siglas en inglés.} con un doble objetivo: tenerlos aislados y permitir cierta granularidad de seguridad sobre ellos \cite{HYGZD2014}. El mecanismo de seguridad para el acceso a estas SS-API de recursos se llama Permisos, tal como se observa en la Figura \ref{fig:ch01:permissions-check}.\\

\begin{figure}[hbtp]
	\begin{center}
		\includegraphics[width=.75\linewidth]{chapter1/permissions_check}
		\caption{Los recursos están resguardados por los permisos \cite{aossec}.}
		\label{fig:ch01:permissions-check}
	\end{center}
\end{figure}
Podemos clasificar los permisos según el riesgo implícito al otorgarlos, resultando las siguientes cuatro categorías \cite{Rom14}:
\begin{itemize}
    \item \emph{Normal:} Son aquellos permisos de bajo riesgo ya que corresponden a características aisladas y son considerados de bajo riesgo para las demás aplicaciones, para el sistema y para el usuario. Son concedidos automáticamente por el sistema, sin solicitar aprobación explícita del usuario.
    \item \emph{Dangerous:} Son aquellos permisos de alto riesgo ya que resguardan los accesos a información sensible para el usuario u otorgan control sobre funcionalidades principales del sistema. Para ser concedidos se requiere aprobación explícita del usuario. 
    \item \emph{Signature:} Son aquellos permisos que son aprobados solamente si la aplicación que los requiere tiene el mismo certificado que la aplicación que los creó. Un certificado identifica a cada desarrollador: establece su identidad y permite que se puedan comprobar las aplicaciones desarrolladas por dicho desarrollador. Cuando el sistema valida el certificado, se otorga el permiso sin requerir aprobación explícita del usuario. Se crearon para permitir que un desarrollador pueda compartir información entre sus distintas aplicaciones sin necesidad de la aprobación del usuario.
    \item \emph{Signature/System:} Son aquellos permisos que controlan el acceso a servicios críticos del sistema. En general, las únicas aplicaciones que los utilizan son las que vienen pre-instaladas en el dispositivo, ya que se utilizan para ciertas situaciones especiales en las que varios proveedores tienen aplicaciones integradas en una imagen del sistema que necesitan compartir funciones específicas explícitamente porque se están creando juntas.
\end{itemize}
\subsection{Manifiesto}
El archivo de manifiesto proporciona información sobre una aplicación al sistema Android, información que el sistema debe tener para poder ejecutar el código de la aplicación. Es por ello que todas las aplicaciones deben tener un archivo \texttt{AndroidManifest.xml} (con ese nombre exacto) en el directorio raíz. En este archivo se declaran todos los componentes que forman parte de la aplicación en cuestión, los permisos que son requeridos (ver sección \ref{ch01-permisos}) y los permisos exportados por la aplicación, entre otras cosas. En la Figura \ref{fig:ch01:manifest} se observa el manifiesto de una aplicación que requiere los permisos \texttt{READ\_CONTACTS} y \texttt{WRITE\_CONTACTS}, entre otros.
\begin{figure}[tbp]
    \centering
    \lstinputlisting[language=XML, basicstyle=\tiny, breaklines=true, frame=single]{AndroidManifest.xml}
    \caption{Manifiesto de una aplicación.}
    \label{fig:ch01:manifest}
\end{figure}
\chapter{Una novedosa forma de crear Apps: Apache Cordova}
\subsection{Overview}
Apache Cordova is an open-source mobile development framework. It allows you to use standard web technologies - HTML5, CSS3, and JavaScript for cross-platform development. Applications execute within wrappers targeted to each platform, and rely on standards-compliant API bindings to access each device's capabilities such as sensors, data, network status, etc.\\
Apache Cordova (formerly PhoneGap) is a mobile application development framework originally created by Nitobi. Adobe Systems purchased Nitobi in 2011, rebranded it as PhoneGap, and later released an open source version of the software called Apache Cordova.[3] Apache Cordova enables software programmers to build applications for mobile devices using CSS3, HTML5, and JavaScript instead of relying on platform-specific APIs like those in Android, iOS, or Windows Phone.[4] It enables wrapping up of CSS, HTML, and JavaScript code depending upon the platform of the device. It extends the features of HTML and JavaScript to work with the device. The resulting applications are hybrid, meaning that they are neither truly native mobile application (because all layout rendering is done via Web views instead of the platform's native UI framework) nor purely Web-based (because they are not just Web apps, but are packaged as apps for distribution and have access to native device APIs). Mixing native and hybrid code snippets has been possible since version 1.9.
\subsection{Architecture}
There are several components to a cordova application. The following diagram shows a high-level view of the cordova application architecture.
https://cordova.apache.org/static/img/guide/cordovaapparchitecture.png
WebView
The Cordova-enabled WebView may provide the application with its entire user interface. On some platforms, it can also be a component within a larger, hybrid application that mixes the WebView with native application components. (See Embedding WebViews for details.)

Web App
This is the part where your application code resides. The application itself is implemented as a web page, by default a local file named index.html, that references CSS, JavaScript, images, media files, or other resources are necessary for it to run. The app executes in a WebView within the native application wrapper, which you distribute to app stores.

This container has a very crucial file - config.xml file that provides information about the app and specifies parameters affecting how it works, such as whether it responds to orientation shifts.

Plugins
Plugins are an integral part of the cordova ecosystem. They provide an interface for Cordova and native components to communicate with each other and bindings to standard device APIs. This enables you to invoke native code from JavaScript.

Apache Cordova project maintains a set of plugins called the Core Plugins. These core plugins provide your application to access device capabilities such as battery, camera, contacts, etc.

In addition to the core plugins, there are several third-party plugins which provide additional bindings to features not necessarily available on all platforms. You can search for Cordova plugins using plugin search or npm. You can also develop your own plugins, as described in the Plugin Development Guide. Plugins may be necessary, for example, to communicate between Cordova and custom native components.

NOTE: When you create a Cordova project it does not have any plugins present. This is the new default behavior. Any plugins you desire, even the core plugins, must be explicitly added.

Cordova does not provide any UI widgets or MV* frameworks. Cordova provides only the runtime in which those can execute. If you wish to use UI widgets and/or an MV* framework, you will need to select those and include them in your application.

\subsection{Plugin Development Guide}
A plugin is a package of injected code that allows the Cordova webview within which the app renders to communicate with the native platform on which it runs. Plugins provide access to device and platform functionality that is ordinarily unavailable to web-based apps. All the main Cordova API features are implemented as plugins, and many others are available that enable features such as bar code scanners, NFC communication, or to tailor calendar interfaces. You can search for available plugins on Cordova Plugin Search page.

Plugins comprise a single JavaScript interface along with corresponding native code libraries for each supported platform. In essence this hides the various native code implementations behind a common JavaScript interface.

This section steps through a simple echo plugin that passes a string from JavaScript to the native platform and back, one that you can use as a model to build far more complex features. This section discusses the basic plugin structure and the outward-facing JavaScript interface. For each corresponding native interface, see the list at the end of this section.
https://cordova.apache.org/plugins/
What is a Cordova plugin?
A plugin is a bit of add-on code that provides JavaScript interface to native components. They allow your app to use native device capabilities beyond what is available to pure web apps.
\subsection{Native and Hybrid apps – A quick overview}
A native app is a smartphone application developed specifically for a mobile operating system (think Objective-C or Swift for iOS vs. Java for Android).

Since the app is developed within a mature ecosystem following the technical and user experience guidelines of the OS (e.g. swipes, app defined gestures, left aligned header on Android, centrally aligned header on iOS, etcetera), it not only has the advantage of faster performance but also “feels right”. What feeling right means is that the in-app interaction has a look and feel consistent with most of the other native apps on the device. The end user is thus more likely to learn how to navigate and use the app faster. Finally, native applications have the significant advantage of being able to easily access and utilize the built-in capabilities of the user’s device (e.g., GPS, address book, camera, etcetera). When a user sends text messages, takes pictures using the device’s default app, set reminders, or uses the device’s music app (the one that came with the phone), they’re using native apps.

In short, native apps are exactly that, native to the user’s OS and hence built per those guidelines.

Hybrid applications are, at core, websites packaged into a native wrapper.

They look and feel like a native app, but ultimately outside of the basic frame of the application (typically restricted to the controls/navigational elements) they are fueled by a company’s website. Basically, a hybrid app is a web app built using HTML5 and JavaScript, wrapped in a native container which loads most of the information on the page as the user navigates through the application (Native apps instead download most of the content when the user first installs the app). Usual suspects here are Facebook, Twitter, Instagram, your mobile banking app, etcetera.
\subsection{Native, HTML5, or Hybrid: Understanding Your Mobile Application Development Options}
https://developer.salesforce.com/page/Native,_HTML5,_or_Hybrid:_Understanding_Your_Mobile_Application_Development_Options
Currently, the Salesforce Mobile SDK supports building three types of apps:
Native apps are specific to a given mobile platform (iOS or Android) using the development tools and language that the respective platform supports (e.g., Xcode and Objective-C with iOS, Eclipse and Java with Android). Native apps look and perform the best.
HTML5 apps use standard web technologies—typically HTML5, JavaScript and CSS. This write-once-run-anywhere approach to mobile development creates cross-platform mobile applications that work on multiple devices. While developers can create sophisticated apps with HTML5 and JavaScript alone, some vital limitations remain at the time of this writing, specifically session management, secure offline storage, and access to native device functionality (camera, calendar, geolocation, etc.)
Hybrid apps make it possible to embed HTML5 apps inside a thin native container, combining the best (and worst) elements of native and HTML5 apps.\\
Native apps are usually developed using an integrated development environment (IDE). IDEs provide tools for building debugging, project management, version control, and other tools professional developers need. While iOS and Android apps are developed using different IDEs and languages, there’s a lot of parity in the development environments, and there’s not much reason to delve into the differences. Simply put, you use the tools required by the device.\\
An HTML5 mobile app is basically a web page, or series of web pages, that are designed to work on a tiny screen. As such, HTML5 apps are device agnostic and can be opened with any modern mobile browser. And because your content is on the web, it's searchable, which can be a huge benefit depending on the app (shopping, for example).An important part of the "write-once-run-anywhere" HTML5 methodology is that distribution and support is much easier than for native apps. Need to make a bug fix or add features? Done and deployed for all users. For a native app, there are longer development and testing cycles, after which the consumer typically must log into a store and download a new version to get the latest fix.
In the last year, HTML5 has emerged as a very popular way for building mobile applications. Multiple UI frameworks are available for solving some of the most complex problems that no developer wants to reinvent. iScroll does a phenomenal job of emulating momentum style scrolling. JQuery Mobile and Sencha Touch provide elegant mobile components, with hundreds if not thousands of plugins that offer everything from carousels to super elaborate controls.However, significant limitations, especially for enterprise mobile, are offline storage and security. While you can implement a semblance of offline capability by caching files on the device, it just isn't a very good solution. Although the underlying database might be encrypted, it’s not as well segmented as a native keychain encryption that protects each app with a developer certificate. Also, if a web app with authentication is launched from the desktop, it will require users to enter their credentials every time the app it is sent to the background. This is a lousy experience for the user. In general, implementing even trivial security measures on a native platform can be complex tasks for a mobile Web developer. Therefore, if security is of the utmost importance, it can be the deciding factor on which mobile technology you choose.\\
Hybrid development combines the best (or worst) of both the native and HTML5 worlds. We define hybrid as a web app, primarily built using HTML5 and JavaScript, that is then wrapped inside a thin native container that provides access to native platform features. PhoneGap is an example of the most popular container for creating hybrid mobile apps.
For the most part, hybrid apps provide the best of both worlds. Existing web developers that have become gurus at optimizing JavaScript, pushing CSS to create beautiful layouts, and writing compliant HTML code that works on any platform can now create sophisticated mobile applications that don’t sacrifice the cool native capabilities. In certain circumstances, native developers can write plugins for tasks like image processing, but in cases like this, the devil is in the details.
On iOS, the embedded web browser or the UIWebView is not identical to the Safari browser. While the differences are minor, they can cause debugging headaches. That’s why it pays off to invest in popular frameworks that have addressed all of the limitations.
